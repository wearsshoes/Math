\documentclass[../../main.tex]{subfiles}

\begin{document}

\ifmainfile
\else
    \ifchapfile
    \else
        \addtocounter{chapter}{3}
        \addtocounter{section}{3}
        \makeatletter
        \begin{flushright}
            \@title \\
            \@author \\
            \@date
        \end{flushright}
    \fi
\fi

\section{Images and inverse images}
\subsection{}
\begin{q}
    Let $f \colon X \to Y$ be a bijective function, and let $f^{-1} \colon Y \to X$ be its inverse. Let $V$ be any subset of $Y$. Prove that the forward image of $V$ under $f^{-1}$ is the same set as the inverse image of $V$ under $f$; thus the fact that both sets are denoted by $f^{-1}(V)$ will not lead to any inconsistency.
\end{q}
    
\begin{proof}
    Suppose $f \colon X \to Y$ is a bijective function, and $f^{-1} \colon Y \to X$ is its inverse, where $V$ is any subset of $Y$.
    Let $f^{-1}(V)$ denote the inverse image of $V$, and let $(f^{-1})(V)$ denote the forward image of $V$ under $f^{-1}$.
    We define
    \begin{equation*}
    \begin{aligned}
        f^{-1}(V) &= \Set{x \in X}{f(x) \in V} \\
        (f^{-1})(V) &= \Set{f^{-1}(y)}{y \in V}
    \end{aligned}
    \end{equation*}
    First we show $f^{-1}(V) \subseteq (f^{-1})(V)$. \\
    Let $z \in f^{-1}(V)$. \\
    Then $z \in X$ and $f(z) \in V$. \\
    Since $f$ is bijective, for all $y$ in $V \subseteq Y$, $y = f(x) = f(f^{-1}(y))$.\\
    Thus $f(z) \in V \implies y \in V$. \\
    Since $f$ is bijective, for all $x$ in $X$, $x = f^{-1}(y) = f^{-1}(f(x))$.\\
    Thus $z \in X \implies z = f^{-1}(y)$. \\




    % The inverse image of $V$ is $f^{-1}(V) = \Set{x \in X}{f(x) \in V}$.
    % Since $f^{-1}$ is the inverse of $f$, it is a function from $Y \to X$. 
    % The forward image of $V$ under $f^{-1}$ is $\Set{f^{-1}(y)}{y \in V}$.
    % With our assumption that $f$ is a bijective function, we know $y=f(x)$, and for all $y$ in $Y$ there exists a unique $x \in X$ such that $f^{-1}(y) = x$.
    % Thus by substitution the forward image of V under $f^{-1}$ is $\Set{x}{f(x) \in V}$.
\end{proof}

\subsection{}
\begin{q}
    Let $f \colon X \to Y$ be a function from one set $X$ to another set $Y$, let $S$ be a subset of $X$, and let $U$ be a subset of $Y$.
\end{q}

\setlist[enumerate]{label = \textbf{\roman*.}}
\begin{enumerate}
    \item
        \begin{q}
            What, in general, can one say about $f^{-1}(f(S))$ and $S$?
        \end{q}
        
        \begin{ans}
            $S$ is a subset of $f^{-1}(f(S))$, but $S$ may not be equal to $f^{-1}(f(S))$.
        \end{ans}
        \begin{proof}
            \emph{(informal)} 
            Let $x$ be an element of $X$. 
            We have $f(S) = \Set{f(x)}{x \in S}$, and therefore $f^{-1}(f(S)) = \Set{x \in X}{f(x) \in f(S)}$.

            Suppose $x \in S$, then $x \in X$ and $f(x) \in f(s)$, thus $x \in f^{-1}(f(S))$ for all $x \in S$, so $S$ is a subset of $f^{-1}(f(S))$. Now instead suppose $x \notin S$. Since we have not stated that $f$ is injective, it is still possible that $f(x) \in f(S)$. Once again $x \in X$ and $f(x) \in f(s)$, thus for some $x$ not in $S$, $x$ may still be in $x \in f^{-1}(f(S))$.
            Thus $f^{-1}(f(S))$ may contain more members of $X$ than $S$ does, so they may not be equal.
            $\square$
        \end{proof}
    \item
        \begin{q}
            What about $f(f^{-1}(U))$ and $U$?
        \end{q}
        
        \begin{ans}
            $f(f^{-1}(U))$ is a subset of $U$, but the two sets may not be equal.
        \end{ans}
            
        \begin{proof}
                \emph{(informal)}
                Let $x$ be an element of $X$. 
                We have $f^{-1}(U)=\Set{x \in X} {f(x) \in U}$. 
                Then $f(f^{-1}(U))=\Set{f(x)}{x \in f^{-1}(U)}$.
                Since $f$ is not stated to be surjective, there may be some $y$ in $U$ for which $y \neq f(x)$ for all $x$.
                So when we take the forward image of $f^{-1}(U)$, every element of $f^{-1}(U)$ is in $U$, but there may be some $y$ in $U$ that are not in $f^{-1}(U)$.
        \end{proof}
    \item 
        \begin{q}
            What about $f^{-1}(f(f^{-1}(U)))$ and $f^{-1}(U)$? 
        \end{q}
        
        \begin{ans}
            
        \end{ans}
            
        \begin{proof}
            \emph{(informal)}
            As before we have $f^{-1}(U) = \Set{x \in X}{f(x) \in U}$, and $f(f^{-1}(U)) = \Set{f(x)}{x \in f^{-1}(U)}$. This means
            \begin{xx}
                \begin{align*}
                    f^{-1}(f(f^{-1}(U))) 
                    &= \Set{x \in X}{f(f^{-1}(U)) \in U}\\
                    &= \Set{x \in X}{\Set{f(x)}{x \in f^{-1}(U)} \in U}\\
                    &= \Set{x \in X}{\Set{f(x)}{x \in \Set{x \in X}{f(x) \in U}}\in U}\\
                    &= (x \in X) \textnormal{ and } (f(x) \textnormal{ is true and } (x \in (x \in X \textnormal{ and } f(x) \in U))\in U). \\
                    &= x \in X \textnormal{ and } f(x) \in U (incomplete)
                \end{align*}
                (good lord...)
            \end{xx}
        \end{proof}
\end{enumerate}

\subsection{}
\begin{q}
    Let $A, B$ be two subsets of a set $X$, and let $f : X \to Y$ be a function. Show that
    \begin{enumerate}
        \item $f(A \cap B) \subseteq f(A) \cap f(B)$,
        \begin{proof}
            We prove this statement by showing every element of $f(A \cap B)$ is an element of $f(A) \cap f(B)$.
            \begin{enumerate}
                \item Let $y$ be an arbitrary element of $f(A \cap B)$. 
                \item $A \subseteq X$ and $B \subseteq X \implies A \cap B \subseteq X$. 
                \item By definition the image of $A \cap B$ under $f$ is $\Set{f(x)}{x \in A \cap B}$. 
                \item By the axiom of replacement (3.7) $y=f(x)$ for some $x \in A \cap B$.
                \item $x \in A \cap B \implies x \in A$ 
                \item $y = f(x)$ for some $x \in A$
                \item $x \in A \cap B \implies x \in B$ 
                \item $y = f(x)$ for some $x \in B$
                \item $y = f(x)$ for some $x \in A$ and $y = f(x)$ for some $x \in B$
                \item $y \in \Set{f(x)}{x \in A}$ and y $\in \Set{f(x)}{x \in B}$
                \item $y \in f(A) \cap f(B)$, as desired.
            \end{enumerate}
        \end{proof}    

        \item $f(A) \setminus f(B) \subseteq f(A \setminus B)$,
        \begin{proof}

            \begin{xx}
                
            \end{xx}
        \end{proof}
        
        \item $f(A \cup B) = f(A) \cup f(B)$.
        \begin{proof}

            \begin{xx}
                We prove this statement by showing every element of $f(A \cup B)$ is an element of $f(A) \cup f(B)$ and vice versa.
                
                Let $y \in (A \cup B)$ be arbitrary.

                Since $A \cup B \subseteq X$,
            \end{xx}
        \end{proof}
        
    \end{enumerate}
    For the first two statements, is it true that the $\subseteq$ relation can be improved to $=$?

    \begin{ans}
    
    \end{ans}
    \begin{proof}
        
    \end{proof}
\end{q}

\addtocounter{subsection}{1}
\subsection{}
\begin{q}
    Let $f \colon X \to Y$ be a function from one set $X$ to another set $Y$. 
    
    \begin{enumerate}
        \item
        Show that $f(f^{-1}(S)) = S \text{ for every } S \subseteq Y \text{ if and only if } f \text{ is surjective.}$
        \begin{proof}
            
            \begin{xx}
                
            \end{xx}
        \end{proof}

        \item 
        Show that \(f^{-1}(f(S)) = S \text{ for every } S \subseteq X \text{ if and only if } f \text{ is injective.}\)
        \begin{proof}
            
            \begin{xx}
                
            \end{xx}
        \end{proof}
    \end{enumerate}

\end{q}

\addtocounter{subsection}{3}
\subsection{}
\begin{q}
    Show that if $\beta$ and $\beta'$ are two elements of a set $I$, and to each $\alpha \in I$ we assign a set $A_\alpha$, then \[
        \{
            x \in A_\beta : x \in A_\alpha \text{ for all } \alpha \in I
        \} = 
        \{
            x \in A_{\beta'} : x \in A_\alpha \text{ for all } \alpha \in I
        \},
    \]
    and so the definition of $\bigcap_{\alpha \in I} A_\alpha$ defined in (3.3) does not depend on $\beta$. 
\end{q}

\begin{proof}
    
    \begin{xx}
        
    \end{xx}
\end{proof}

\begin{q}
    Also explain why (3.4) is true.
\end{q}

\begin{proof}
    
    \begin{xx}
        
    \end{xx}
\end{proof}

\subsection{}
\begin{q}
    Suppose that $I$ and $J$ are two sets, and for all $\alpha \in I \cup J$ let $A_\alpha$ be a set. Show that
    \[\bigcup_{\alpha \in I} A_\alpha \cup \bigcup_{\alpha \in J} A_\alpha = \bigcup_{\alpha \in I \cup J} A_\alpha.\]
\end{q}

    \begin{proof}
    
        \begin{xx}
            
        \end{xx}
    \end{proof}

\begin{q}
    If $I$ and $J$ are non-empty, show that
    \[\bigcap_{\alpha \in I} A_\alpha \cap \bigcap_{\alpha \in J} A_\alpha = \bigcap_{\alpha \in I \cup J} A_\alpha.\]
\end{q}


\begin{proof}
    
    \begin{xx}
        
    \end{xx}
\end{proof}

\end{document}