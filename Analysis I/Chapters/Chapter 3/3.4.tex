\documentclass[../../main.tex]{subfiles}

\begin{document}
\exerciseset{3}{4}{Images and inverse images}

\subsection{}
\begin{q}
    Let $f \colon X \to Y$ be a bijective function, and let $f^{-1} \colon Y \to X$ be its inverse. Let $V$ be any subset of $Y$. Prove that the forward image of $V$ under $f^{-1}$ is the same set as the inverse image of $V$ under $f$; thus the fact that both sets are denoted by $f^{-1}(V)$ will not lead to any inconsistency.
\end{q}
    
\begin{proof}
    Suppose $f \colon X \to Y$ is a bijective function, and $f^{-1} \colon Y \to X$ is its inverse, where $V$ is any subset of $Y$.
    Let $f^{-1}(V)$ denote the inverse image of $V$, and let $(f^{-1})(V)$ denote the forward image of $V$ under $f^{-1}$.
    We define
    \begin{equation*}
    \begin{aligned}
        & f^{-1}(V) = \Set{x \in X}{f(x) \in V} \\
        & (f^{-1})(V) = \Set{f^{-1}(y)}{y \in V}
    \end{aligned}
    \end{equation*}

    \begin{linebyline}
            \item First we show $f^{-1}(V) \subseteq (f^{-1})(V)$.
            \item Let $z \in f^{-1}(V)$. 
            \item Then $z \in X$ and $f(z) \in V$. 
            \item Since $f$ is bijective, for all $y$ in $V \subseteq Y$, $y = f(x) = f(f^{-1}(y))$.
            \item Thus $f(z) \in V \implies y \in V$. 
            \item Since $f$ is bijective, for all $x$ in $X$, $x = f^{-1}(y) = f^{-1}(f(x))$.
            \item Thus $z \in X \implies z = f^{-1}(y)$. 
    \end{linebyline}
\end{proof}

\subsection{}
\begin{q}
    Let $f \colon X \to Y$ be a function from one set $X$ to another set $Y$, let $S$ be a subset of $X$, and let $U$ be a subset of $Y$.
\end{q}

\begin{enumerate}
    \item
        \begin{q}
            What, in general, can one say about $f^{-1}(f(S))$ and $S$?
        \end{q}
        
        \begin{ans}
            $S$ is a subset of $f^{-1}(f(S))$, but $S$ may not be equal to $f^{-1}(f(S))$.
        \end{ans}
        \begin{proof}
            \emph{(informal)} 
            Let $x$ be an element of $X$. 
            We have $f(S) = \Set{f(x)}{x \in S}$, and therefore $f^{-1}(f(S)) = \Set{x \in X}{f(x) \in f(S)}$.

            Suppose $x \in S$, then $x \in X$ and $f(x) \in f(s)$, thus $x \in f^{-1}(f(S))$ for all $x \in S$, so $S$ is a subset of $f^{-1}(f(S))$. Now instead suppose $x \notin S$. Since we have not stated that $f$ is injective, it is still possible that $f(x) \in f(S)$. Once again $x \in X$ and $f(x) \in f(s)$, thus for some $x$ not in $S$, $x$ may still be in $x \in f^{-1}(f(S))$.
            Thus $f^{-1}(f(S))$ may contain more members of $X$ than $S$ does, so they may not be equal.
            $\square$
        \end{proof}
    \item
        \begin{q}
            What about $f(f^{-1}(U))$ and $U$?
        \end{q}
        
        \begin{ans}
            $f(f^{-1}(U))$ is a subset of $U$, but the two sets may not be equal.
        \end{ans}
            
        \begin{proof}
                \emph{(informal)}
                Let $x$ be an element of $X$. 
                We have $f^{-1}(U)=\Set{x \in X} {f(x) \in U}$. 
                Then $f(f^{-1}(U))=\Set{f(x)}{x \in f^{-1}(U)}$.
                Since $f$ is not stated to be surjective, there may be some $y$ in $U$ for which $y \neq f(x)$ for all $x$.
                So when we take the forward image of $f^{-1}(U)$, every element of $f^{-1}(U)$ is in $U$, but there may be some $y$ in $U$ that are not in $f^{-1}(U)$.
        \end{proof}
    \item 
        \begin{q}
            What about $f^{-1}(f(f^{-1}(U)))$ and $f^{-1}(U)$? 
        \end{q}
        
        \begin{ans}
            
        \end{ans}
            
        \begin{proof}
            \emph{(informal)}
            As before we have $f^{-1}(U) = \Set{x \in X}{f(x) \in U}$, and $f(f^{-1}(U)) = \Set{f(x)}{x \in f^{-1}(U)}$.
            \begin{align*}
                f^{-1}(f(f^{-1}(U))) 
                &= \Set{x \in X}{f(f^{-1}(U)) \in U}\\
                &= x \in X \textnormal{ and } f(f^{-1}(U)) \in U\\
                &= x \in X \textnormal{ and } \Set{f(x)}{x \in f^{-1}(U)} \in U\\
                &= x \in X \textnormal{ and } ( \exists x \textnormal{ such that } y=f(x) \textnormal{ and } x \in f^{-1}(U) ) \in U\\  
            \end{align*}
            \begin{xx}
                \begin{align*}
                    f^{-1}(f(f^{-1}(U))) 
                    &= \Set{x \in X}{f(f^{-1}(U)) \in U}\\
                    &= \Set{x \in X}{\Set{f(x)}{x \in f^{-1}(U)} \in U}\\
                    &= \Set{x \in X}{\Set{f(x)}{x \in \Set{x \in X}{f(x) \in U}}\in U}\\
                    &= (x \in X) \textnormal{ and } (f(x) \textnormal{ is true and } (x \in (x \in X \textnormal{ and } f(x) \in U))\in U). \\
                    &= x \in X \textnormal{ and } f(x) \in U (incomplete)
                \end{align*}
                (good lord...)
            \end{xx}
        \end{proof}
\end{enumerate}

\subsection{}
\begin{q}
    Let $A, B$ be two subsets of a set $X$, and let $f : X \to Y$ be a function. Show that
    \begin{enumerate}
        \item $f(A \cap B) \subseteq f(A) \cap f(B)$,
        \begin{proof}
            We prove this statement by showing every element of $f(A \cap B)$ is an element of $f(A) \cap f(B)$.
            \begin{linebyline}
                \item Let $y$ be an arbitrary element of $f(A \cap B)$. 
                \item $A \subseteq X$ and $B \subseteq X \implies A \cap B \subseteq X$. 
                \item By definition the image of $A \cap B$ under $f$ is $\Set{f(x)}{x \in A \cap B}$. 
                \item By the axiom of replacement (3.7) $y=f(x)$ for some $x \in A \cap B$.
                \item $x \in A \cap B \implies x \in A$ 
                \item $y = f(x)$ for some $x \in A$
                \item $x \in A \cap B \implies x \in B$ 
                \item $y = f(x)$ for some $x \in B$
                \item $y = f(x)$ for some $x \in A$ and $y = f(x)$ for some $x \in B$
                \item $y \in \Set{f(x)}{x \in A}$ and y $\in \Set{f(x)}{x \in B}$
                \item $y \in f(A) \cap f(B)$, as desired.
            \end{linebyline}
        \end{proof}    

        \item $f(A) \setminus f(B) \subseteq f(A \setminus B)$,
        \begin{proof}
            We prove this statement by showing every element of $f(A) \setminus f(B)$ is an element of $f(A \setminus B)$.
            \begin{linebyline}
                \item Let $y \in f(A) \setminus f(B)$ be arbitrary. 
                \why{Conditional introduction}
                \item $y \in f(A)$ and $y \notin f(B)$. 
                \item $\exists x \in A \; y = f(x)$
                \item Suppose $x$ such that $x \in A$ and $y = f(x)$
                \begin{linebyline}
                    \item $x \in A$
                    \item $y = f(x)$
                    \item $\forall z \in B \; y \neq f(z)$
                    \item $\forall z \; z \in B \implies y \neq f(z)$
                    \item $\forall z \; y = f(z) \implies z \notin B$

                    \item $y = f(x) \implies x \notin B$
                    \item $x \notin B$
                    \item $ x \in A$, $x \notin B$, and $y = f(x)$.
                    \item $y = f(x)$ and $x \in A \setminus B$.
                    \item $y \in \Set{y}{y = f(x)$ for $x \in A \setminus B}$.
                \end{linebyline}
                \item $y \in f(A \setminus B)$ \why{Existential elimination}
                \item $y \in f(A) \setminus f(B) \implies y \in f(A \setminus B)$ \why{Conditional elimination}
            \end{linebyline}
            Thus $f(A) \setminus f(B) \subseteq f(A \setminus B)$.
        \end{proof}
        
        \item $f(A \cup B) = f(A) \cup f(B)$.
        \begin{proof}
            We prove this statement by showing every element of $f(A \cup B)$ is an element of $f(A) \cup f(B)$ and vice versa.
            First we do the forward direction:
            \begin{linebyline}
                \item Let $y \in f(A \cup B)$ be arbitrary.
                \item $A \in X$
                \item $B \in X$
                \item $A \cup B \in X$
                \item $y \in \Set{f(x)}{x \in A \cup B}$
                \item $\exists x$ such that $x \in A \cup B$ and $y=f(x)$
                \item Suppose $x$ such that $x \in A \cup B$ and $y=f(x)$
                \begin{linebyline}
                    \item $y = f(x)$
                    \item $x \in A \cup B$
                    \item $x \in A$ or $x \in B$
                    \item ($x \in A$ and $y=f(x)$) or ($x \in B$ and $y=f(x)$)
                    \begin{linebyline}
                        \item{test}
                    \end{linebyline}
                    \item $y \in \Set{f(x)}{x \in A}$ or $y \in \Set{y=f(x)}{x \in B}$
                    \item $y \in f(A)$ or $y \in f(B)$
                    \item $y \in f(A) \cup f(B)$
                \end{linebyline}
                \item $y \in f(A \cup B) \implies y \in f(A) \cup f(B)$
                \item $f(A \cup B) \subseteq f(A) \cup f(B)$
            \end{linebyline}
            Now in the backwards direction.
            \begin{linebyline}
                \item Let $y \in f(A) \cup f(B)$ be arbitrary.
                \item $y \in f(A)$ or $y \in f(B)$
                \item Case $y \in f(A)$
                \begin{linebyline}
                    \item $y \in \Set{f(x)}{x \in A}$ 
                    \item $\exists x$ such that ($x \in A$ and $y=f(x)$)
                    \item Suppose $x$ such that ($x \in A$ and $y=f(x)$) 
                    \begin{linebyline}
                        \item $x \in A$ and $y=f(x)$
                    \end{linebyline}
                \end{linebyline}
                \item Case $y \in f(B)$
                \begin{linebyline}
                    \item $y \in \Set{y=f(x)}{x \in B}$
                    \item $\exists x$ such that ($x \in B$ and $y=f(x)$)
                    \item Suppose $x$ such that ($x \in B$ and $y=f(x)$)
                    \begin{linebyline}
                        \item $x \in B$ and $y=f(x)$
                    \end{linebyline}
                \end{linebyline}
                \item $(x \in B$ and $y=f(x))$ or ($x \in A$ and $y=f(x)$) 
                \item $y=f(x)$ and ($x \in A$ or $x \in B$) \why{Distributivity}
                \item $y=f(x)$ and ($x \in A \cup B$)
                \item $y \in \Set{f{x}}{x \in A \cup B}$
                \item $y \in f(A) \cup f(B) \implies y \in \Set{f(x)}{x \in A \cup B}$
                \item $f(A) \cup f(B) \subseteq f(A \cup B)$
            \end{linebyline}
            Thus we have $f(A \cup B) = f(A) \cup f(B)$.
        \end{proof}
        
    \end{enumerate}
    For the first two statements, is it true that the $\subseteq$ relation can be improved to $=$?

    \begin{ans}
    
    \end{ans}
    \begin{proof}
        I want to first try to prove $f(A \cap B) = f(A) \cap f(B)$. Since I already have $f(A \cap B) \subseteq f(A) \cap f(B)$, I just need $f(A) \cap f(B) \subseteq f(A \cap B)$.
        
        \begin{linebyline}
            \item Suppose $y \in f(A) \cap f(B)$
            \item $y \in f(A)$ and $y \in f(B)$
            \item $y \in \Set{f(x)}{x \in A}$
            \item $\exists x$ st. $y = f(x)$ and $x \in A$
            \item Suppose $x$ st. $y = f(x)$ and $x \in A$
            \item $y \in \Set{f(x)}{x \in B}$
            \item $\exists x$ st. $y = f(x)$ and $x \in A$
        \end{linebyline}

        Next I'm going to try to prove $f(A) \setminus f(B) = f(A \setminus B)$. I already have $f(A) \setminus f(B) \subseteq f(A \setminus B)$ and I just need $f(A \setminus B) \subseteq f(A) \setminus f(B)$.
        \begin{linebyline}
            \item Suppose $y \in f(A \setminus B)$.
            \item $\exists x$ such that $y=f(x)$ and $x \in A \setminus B$.
            \item Suppose $x$ such that $y=f(x)$ and $x \in A \setminus B$.
            \begin{linebyline}
                \item $y=f(x)$
                \item $x \in A \setminus B$
                \item $x \in A$ and $x \notin B$
                \item $y=f(x)$ and $x \in A$
                \item $y \in \Set{f(x)}{x \in A}$
                \item $y \in f(A)$
                \item $y=f(x)$ and $x \notin B$
                \item \sout{$y \in \Set{f(x)}{x \notin B}$} (not useful!)
            \end{linebyline}
        \end{linebyline}
        \begin{xx}
            not sure where to go from here
        \end{xx}
    \end{proof}
\end{q}

\addtocounter{subsection}{1}
\subsection{}
\begin{q}
    Let $f \colon X \to Y$ be a function from one set $X$ to another set $Y$. 
    
    \begin{enumerate}
        \item
        Show that $f(f^{-1}(S)) = S \text{ for every } S \subseteq Y \text{ if and only if } f \text{ is surjective.}$
        \begin{proof}
            
            \begin{xx}
                
            \end{xx}
        \end{proof}
        \item 
        Show that \(f^{-1}(f(S)) = S \text{ for every } S \subseteq X \text{ if and only if } f \text{ is injective.}\)
        \begin{proof}
            
            \begin{xx}
                
            \end{xx}
        \end{proof}
    \end{enumerate}

\end{q}

\subsection{}
\begin{enumerate}
    \item
        \begin{q}
            Prove Lemma 3.4.10. (Hint: start with the set $\{0, 1\}^X$ and apply the replacement axiom, replacing each function $f$ with the object $f^{-1}(\{1\})$. See also Exercise 3.5.11.)

            (Lemma 3.4.10):
            Let $X$ be a set. Then the set $\Set{Y}{Y$ is a subset of $X}$ is a set. That is to say, there exists a set $Z$ such that $Y \in Z \iff Y \subseteq X$.
        \end{q}
        \begin{proof}                                    
        We need to prove $\Set{Y}{Y \subseteq X}$ exists. We construct a set that we know exists, and then prove that it is equal to $\Set{Y}{Y \subseteq X}$.
        \begin{linebyline}
            \item Let $X$ be an arbitrary set.
            \item Let $F$ be the set $\{0, 1\}^{X}$ 
            \item $f \in F \iff f : X \to \{0,1\}$ \why{power set axiom}
            \item Let $P(f,a)$ such that $P(f,a) \iff a=f^{-1}(\{1\})$
            \item Let $Z$ be the set $\Set{a}{P(f, a)$ is true for some $f \in F}$
            \item $z \in Z \iff P(f, z) $ is true for some $f \in F$. \why{axiom of replacement}
            \item Suppose $Y \in Z$.
            \begin{linebyline}
                \item $P(f, Y)$ is true for some $f \in F$ \why{5}
                \item Suppose $g \in F$ such that $P(g, Y)$
                \begin{linebyline}
                    \item $Y=g^{-1}(\{1\})$ \why{3}
                    \item $Y \subseteq X$ \why{def of inverse image}
                \end{linebyline}
                \item $Y \subseteq X$
            \end{linebyline}
            \item $\forall Y, \; Y \in Z \implies Y \subseteq X$
            \item Suppose $Y \subseteq X$.
            \begin{linebyline}
                \item Let $g: X \to \{0, 1\} = x \mapsto 1$ for $x \in Y$ and $x \mapsto 0$ otherwise. 
                \item $g \in F$ \why{3}
                \item $Y = g^{-1}(\{1\})$ \why{7.5.1}
                \item $P(g, Y)$ \why{4}
                \item $Y \in Z$ \why{6}
            \end{linebyline}
            \item $\forall Y, \; Y \subseteq X \implies Y \in Z$
            \item $\forall z, \; z \in Z \iff z \in \Set{Y}{Y \subseteq X}$. \why{axiom of set equality}
            \item $Z = \Set{Y}{Y \subseteq X}$
            \item $\Set{Y}{Y \subseteq X}$ is a set.
        \end{linebyline}

    Let X be arbitrary. 
    We need to prove $\Set{Y}{Y \subseteq X}$ exists. 
    Let's proceed by creating a valid set and proving it is equal to $\Set{Y}{Y \subseteq X}$.
    Let F be the power set $\{0,1\}^X$. 
    Let $Z$ be the set $\Set{a}{P(f, a)$ is true for some $f \in F}$, where $P(f,a)$ is $a = f^{-1}(\{1\})$. 
    In order to show $Z=\Set{Y}{Y \subseteq X}$, we need to show $z \in Z \iff z \subseteq X$ is true.
    We will first prove the forward direction

    Suppose $Y \in Z$. 
    Then by the axiom of specification $P(f, Y)$ is true for some $f \in F$. 
    Choose $g \in F$ such that $P(g, Y)$ is true. 
    Then $Y = g^{-1}(\{1\})$.
    Since by definition $g^{-1}(\{1\})$ is in $X$, $Y \in X$.

    Conversely suppose $Y \subseteq X$.
    Let $g : X \to \{0, 1\}$ be the function $x \mapsto 1$ for $x \in Y$ and $x \mapsto 0$ otherwise. Since every function $X \to \{0, 1\}$ is in $F$, $g$ is in $F$. Since $x \mapsto 1$ for $x \in Y$, we know $Y = g^{-1}(\{1\})$, which means $P(g, Y)$ is true, and thus $Y \in Z$.

    We've proved $Y \in Z \iff Y \subseteq X$, so $Z=\Set{Y}{Y \subseteq X}$ is true, and since $Z$ exists, $\Set{Y}{Y \subseteq X}$ exists.
    \end{proof}

    \item 
    \begin{q}
        Conversely, show that Axiom 3.11 can be deduced from the preceding axioms of set theory if one accepts Lemma 3.4.10 as an axiom. (This may help explain why we refer to Axiom 3.11 as the "power set axiom".)
    \end{q}
    
    \begin{proof}
            
    \end{proof}        
\end{enumerate} 

\subsection{}
\begin{q}
    Let $X$, $Y$ be sets. Define a partial function from $X$ to $Y$ to be any function $f : X' \to Y'$ whose domain $X'$ is a subset of $X$, and whose codomain $Y'$ is a subset of $Y$. Show that the collection of all partial functions from $X$ to $Y$ is itself a set. (Hint: use Exercise 3.4.6, the power set axiom, the replacement axiom, and the union axiom.
\end{q}

\begin{proof}
    We wish to show that $Y'^{X'}$ defined as $\Set{f'}{f' : X' \to Y'}$ exists.
\begin{linebyline}
    \item Let $X$, $Y$ be arbitrary sets.
    \item For $X' \subseteq X$ and $Y' \subseteq Y, \; f' : X' \to Y'$
    \item $\Set{f}{f : X \to Y}$ is a set. \why{power set axiom}
    \item $\Set{X'}{X' \subseteq X}$ is a set. \why{Lemma 3.4.10, see above}
    \item $\Set{Y'}{Y' \subseteq Y}$ is a set. \why{Lemma 3.4.10, see above}
    \item 
    \item $\Set{f'}{f': X' \to Y}$ is a set.
    \item $\Set{f'}{f': X \to Y'}$ is a set.
    \item $\Set{f'}{f': X' \to Y} \cup \Set{f'}{f': X \to Y'}$ is a set. \why{union axiom}
    \item $\forall a, \; a \in \Set{f'}{f': X' \to Y} \cup \Set{f'}{f': X \to Y'} \iff a \in X' \to Y$ and $a \in X \to Y'$\why{axiom of replacement}
    \item $a \in X' \to Y'$
    \item $a \in \Set{f'}{f' : X' \to Y'}$ \why{axiom of replacement}
    \item $\Set{f'}{f' : X' \to Y'}$ is a set.
\end{linebyline}    
\end{proof}
    \begin{xx}
    not proved :(     
    \end{xx}


\addtocounter{subsection}{1}
\subsection{}
\begin{q}
    Show that if $\beta$ and $\beta'$ are two elements of a set $I$, and to each $\alpha \in I$ we assign a set $A_\alpha$, then \[
        \{
            x \in A_\beta : x \in A_\alpha \text{ for all } \alpha \in I
        \} = 
        \{
            x \in A_{\beta'} : x \in A_\alpha \text{ for all } \alpha \in I
        \},
    \]
    and so the definition of $\bigcap_{\alpha \in I} A_\alpha$ defined in (3.3) does not depend on $\beta$. 
\end{q}

\begin{proof}
    
    \begin{xx}
        
    \end{xx}
\end{proof}

\begin{q}
    Also explain why (3.4) is true.
\end{q}

\begin{proof}
    
    \begin{xx}
        
    \end{xx}
\end{proof}

\subsection{}
\begin{q}
    Suppose that $I$ and $J$ are two sets, and for all $\alpha \in I \cup J$ let $A_\alpha$ be a set. Show that
    \[\bigcup_{\alpha \in I} A_\alpha \cup \bigcup_{\alpha \in J} A_\alpha = \bigcup_{\alpha \in I \cup J} A_\alpha.\]
\end{q}

    \begin{proof}
    We need to show that every element of $\bigcup_{\alpha \in I} A_\alpha \cup \bigcup_{\alpha \in J} A_\alpha$ is also in $\bigcup_{\alpha \in I \cup J} A_\alpha$ and vice versa. We begin in the forward direction.
        \begin{xx}
        \begin{linebyline}
            \item
        \end{linebyline}
    Now in reverse:
        \begin{linebyline}
            \item
        \end{linebyline}
        \end{xx}
    \end{proof}

\begin{q}
    If $I$ and $J$ are non-empty, show that
    \[\bigcap_{\alpha \in I} A_\alpha \cap \bigcap_{\alpha \in J} A_\alpha = \bigcap_{\alpha \in I \cup J} A_\alpha.\]
\end{q}


\begin{proof}
    
    \begin{xx}
    \begin{linebyline}
        
    \end{linebyline}
    \end{xx}
\end{proof}

\end{document}