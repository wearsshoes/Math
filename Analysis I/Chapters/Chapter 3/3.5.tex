\documentclass[../../main.tex]{subfiles}

\begin{document}
\exerciseset{3}{5}{Cartesian Products}

\addtocounter{subsection}{1}
\subsection{}
\begin{q}
    Suppose we define an ordered $n$-tuple to be a surjective function $x : \Set{i \in \mathbb{N}}{1 \leq i \leq n} \to X$ whose codomain is some arbitrary set $X$ (so different ordered $n$-tuples are allowed to have different ranges); we then write $x_i$ for $x(i)$ and also write $x$ as $(x_i)_{1 \leq i \leq n}$. Using this definition, verify that we have $(x_i)_{1 \leq i \leq n} = (y_i)_{1 \leq i \leq n}$ if and only if $x_i = y_i$ for all $1 \leq i \leq n$. 
\end{q}

\begin{proof}
    We are asked to prove that the n-tuples $x$ and $y$ are equal for all $1 \leq i \leq n$ by showing that their components $x_i$ and $y_i$ are equal for all  $1 \leq i \leq n$, and vice versa. We rewrite our definitions for clarity:
    \begin{lxl}
        \item Let $X$ be any set. 
        \item Let $x : \Set{i \in \mathbb{N}}{1 \leq i \leq n} \to X$
        \item Let $x_i = x(i)$ 
        \item Let $x = (x_i)_{1 \leq i \leq n}$
    \end{lxl}
    We proceed in the forward direction.
    \begin{lxl}[resume]
        \item Suppose $(x_i)_{1 \leq i \leq n} = (y_i)_{1 \leq i \leq n}$.
        \begin{lxl}
            \item $x = y$ \why{line 4}
            \item Suppose ${1 \leq i \leq n}$
            \begin{lxl}
                \item $i \in \Set{i \in \mathbb{N}}{1 \leq i \leq n}$
                \item $x(i)$=$y(i)$ \why{line 5.1}
                \item $x_i=y_i$ \why{line 3}
            \end{lxl}
            \item $\forall i \in \mathbb{N} \quad {1 {\leq} i {\leq} n} \implies x_i=y_i$
        \end{lxl}
        \item $(x_i)_{1 \leq i \leq n} = (y_i)_{1 \leq i \leq n} \implies \bigl( \forall i \in \mathbb{N} \quad {1 {\leq} i {\leq} n} \implies x_i=y_i \bigr)$.
    \end{lxl}
    Now in the reverse direction:
    \begin{lxl}[resume]
        \item Suppose $i \in \mathbb{N}$ st. ${1 {\leq} i {\leq} n} \implies x_i=y_i$.
        \begin{lxl}
            \item $i \in \Set{i \in \mathbb{N}}{1 \leq i \leq n}$ \why{line 2}
            \item $x_i=y_i$ \why{slightly awkward MP}
            \item $(x)_{1 \leq i \leq n}(i)=y_{1 \leq i \leq n}(i)$ \why{lines 3, 4}
        \end{lxl}
        \item $\forall i \in \mathbb{N} \quad \bigl({1 {\leq} i {\leq} n} \implies x_i=y_i \bigr) \implies (x)_{1 \leq i \leq n}(i)=y_{1 \leq i \leq n}(i)$ 
    \end{lxl}\item
    Thus they are equal.
\end{proof}
\begin{xx}

\end{xx}

\begin{q}
    Also, show that if $(X_i){1 \leq i \leq n}$ are an ordered $n$-tuple of sets, then the Cartesian product, as defined in Definition 3.5.6, is indeed a set. (Hint: use Exercise 3.4.7 and the axiom of specification.)
\end{q}

\begin{proof}
\begin{lxl}
    \item Let 
    \item Suppose $(X_i){1 \leq i \leq n}$.
    \begin{lxl}
        \item
    \end{lxl}
\end{lxl}
\end{proof}
\begin{xx}
    Unsure how to approach.
\end{xx}

\addtocounter{subsection}{1}
\subsection{}
\begin{q}
    Let $A, B, C$ be sets. Show that: 
\end{q}
    \begin{enumerate}
        \begin{q}
        \item $A \times (B \cup C) = (A \times B) \cup (A \times C)$,
        \end{q}
        \begin{proof}
            We need to show every element of $A \times (B \cup C)$ is also in $(A \times B) \cup (A \times C)$ and vice versa. We first prove the forward direction:
            \begin{lxl}
                \item Suppose $a \in A \times (B \cup C)$
                \begin{lxl}
                    \item $a \in \Set{(x,y)}{x \in A$ and $y \in (B \cup C)}$
                    \item $a \in \Set{(x,y)}{x \in A$ and ($y \in B$ or $y \in C)}$
                    \item $a \in \Set{(x,y)}{x \in A$ and ($y \in B$ or $y \in C)}$
                    \item $a \in \Set{(x,y)}{(x \in A$ and $y \in B)$ or $(x \in A$ and $y \in C)}$
                    \item $a = (x,y)$ and $\bigl( (x \in A$ and $y \in B)$ or $(x \in A$ and $y \in C) \bigr)$ for some $x \in A$ and some $y \in B$
                    \item Let $x$ and $y$ st. $a=(x,y)$
                    \item \begin{lxl}
                        \item $(x \in A$ and $y \in B)$ or $(x \in A$ and $y \in C)$
                        \item Case $(x \in A$ and $y \in B)$
                        \begin{lxl}
                            \item $a \in \Set{(x,y)}{x \in A$ and $y \in B}$
                            \item $a \in (A \times B)$
                        \end{lxl}
                        \item Case $(x \in A$ and $y \in C)$
                        \begin{lxl}
                            \item $a \in \Set{(x,y)}{x \in A$ and $y \in C}$
                            \item $a \in (A \times C)$
                        \end{lxl}
                        \item $a \in (A \times B)$ or $a \in (A \times C)$
                        \item $a \in (A \times B) \cup (A \times C)$
                    \end{lxl}
                \end{lxl}
                \item $a \in A \times (B \cup C) \implies a \in (A \times B) \cup (A \times C)$
                \item $A \times (B \cup C) \subseteq (A \times B) \cup (A \times C)$
            \end{lxl}
            \item Now this is mostly invertible:
            \begin{lxl}[resume]
                \item Suppose $a \in (A \times B) \cup (A \times C)$
                \begin{lxl}
                    \item $a \in (A \times B)$ or $a \in (A \times C)$
                    \item Case $a \in (A \times B)$
                    \begin{lxl}
                        \item $a \in \Set{(x,y)}{x \in A$ and $y \in B}$
                        \item $\exists x \in A, y \in B $ st. $a = (x,y)$
                        \item Let $x, y$ st. $a = (x,y)$
                        \begin{lxl}
                            \item $(x \in A$ and $y \in B$)
                            \item $(x \in A$ and $x \in B)$ or $(x \in A$ and $x \in C)$
                            \item $a = (x,y)$ and $\bigl( (x \in A$ and $x \in B)$ or $(x \in A$ and $y \in C) \bigr)$
                            \item $a \in \Set{(x,y)}{(x \in A$ and $y \in B)$ or $(x \in A$ and $y \in C)}$

                        \end{lxl}
                    \end{lxl}
                    \item Case $a \in (A \times C)$
                    \begin{lxl}
                        \item $a \in \Set{(x,y)}{x \in A$ and $y \in C}$
                        \item $\exists x \in A, y \in B $ st. $a = (x,y)$
                        \item Let $x, y$ st. $a = (x,y)$
                        \begin{lxl}
                            \item $(x \in A$ and $y \in C)$
                            \item $(x \in A$ and $x \in B)$ or $(x \in A$ and $x \in C)$
                            \item $a = (x,y)$ and $\bigl( (x \in A$ and $x \in B)$ or $(x \in A$ and $y \in C) \bigr)$
                            \item $a \in \Set{(x,y)}{(x \in A$ and $y \in B)$ or $(x \in A$ and $y \in C)}$

                        \end{lxl}
                    \end{lxl}
                    \item $a \in \Set{(x,y)}{(x \in A$ and $y \in B)$ or $(x \in A$ and $y \in C)}$ \why{4.2.3.4 and 4.3.3.4}
                    \item $a \in \Set{(x,y)}{x \in A$ and $(y \in B$ or $y \in C)}$
                    \item $a \in \Set{(x,y)}{x \in A$ and $y \in (B \cup C)}$
                    \item $a \in A \times (B \cup C)$
                \end{lxl}
                \item $a \in (A \times B) \cup (A \times C) \implies a \in A \times (B \cup C)$
                \item $(A \times B) \cup (A \times C) \subseteq A \times (B \cup C)$
            \end{lxl}
            And thus we have $A \times (B \cup C) = (A \times B) \cup (A \times C)$
        \end{proof}
        \begin{xx}
            
        \end{xx}    
        \item \begin{q}
$A \times (B \cap C) = (A \times B) \cap (A \times C)$,
        \end{q}    
        \begin{proof}
            [Skipped] We need to show  that every element of $A \times (B \cap C)$  is in $(A \times B) \cap (B \times A)$, and vice versa. First we prove the forward direction. Let $(a_1, a_2)$, such that $a_1 \in A$ and $( a_2 \in B$ and $a_2 \in C )$. So we can distribute this such that ($a_1 \in A $ and $a_2 \in B$) and ($a_1 \in A$ and $a_2 \in C$). Thus every element of $A \times (B \cap C)$ is in $(A \times B) \cap (A \times C)$. It remains to show the truth of the other direction. 
            
            Instead let $a \in (A \times B) \cap (A \times C)$. We have $a \in (A \times B)$ and $a \in (A \times C)$. Then $a_1 \in A$ and $a_2 \in B$ and $a_2 \in C$ as well. Then $a_2 \in B \cap C$, so $a \in A \times (B \cap C)$, completing our proof. 
        \end{proof}
        \begin{xx}
            
        \end{xx}      
        \item \begin{q}
            and $A \times (B\setminus C) = (A \times B)\setminus(A \times C)$. (One can of course prove similar identities in which the roles of the left and right factors of the Cartesian product are reversed.)
        \end{q}
        \begin{proof}
            We want to first prove every element of A X (B - C) is in (A X B) - (A X C) and vice versa. We begin in the forward direction. 
            Let a in A X (B - C). 
            Then for the ordered pair (a1, a2), a1 is some element in A and a2 is some element in (B - C).
            a2 is in B, so (a1, a2) is in A X B
            a2 is not in C, so (a1, a2) is not in A X C.
            Then (a1, a2) is in A X B and not in A X C.
            So (a1, a2) is in (A X B) - (A x C)
            a is in (A X B) - (A x C)
            Now in the other direction
            Let a in (A X B) - (A X C)
            a in (A X B) and a not in (A X C)
            (a1 is some element in A and a2 is some element in B) and (not (a1 is some element in A and a2 is some element in C))
            (a1 is some element in A and a2 is some element in B) and ((a1 is not in A) or a2 not in C)
            since a1 in A, a2 must not be in C.
            a1 in A, a2 in B, a2 not in C 
            a1 in A a2 in (B - C)
            a in A X (B - C)
            We have proved both directions
        \end{proof}    
    \end{enumerate}    

\addtocounter{subsection}{2}
\subsection{}
\begin{q}
    Let $X$ and $Y$ be sets, and let $\pi_{X\times Y\to X} : X \times Y \to X$ and $\pi_{X\times Y\to Y} : X \times Y \to Y$ be the maps $\pi_{X\times Y\to X}(x, y) := x$ and $\pi_{X\times Y\to Y}(x, y) := y$; these maps are known as the coordinate functions on $X\times Y$. Show that for any functions $f : Z \to X$ and $g : Z \to Y$, there exists a unique function $h : Z \to X \times Y$ such that $\pi_{X\times Y\to X} \circ h = f$ and $\pi_{X\times Y\to Y} \circ h = g$. (Compare this to the last part of Exercise 3.3.8, and to Exercise 3.1.7.) This function $h$ is known as the pairing of $f$ and $g$ and is denoted $h = (f, g)$.
\end{q}

\begin{proof}
    Rewrite the definitions:
    \begin{lxl}
        \item Let $X$ and $Y$ be arbitrary sets
        \item Let $\pi_{X\times Y\to X} : X \times Y \to X$ be the map $\pi_{X\times Y\to X}(x, y) := x$
        \item Let $\pi_{X\times Y\to Y} : X \times Y \to Y$ be the map $\pi_{X\times Y\to Y}(x, y) := y$
        \item Let $f : Z \to X$ and $g : Z \to Y$ be arbitrary functions.
    \end{lxl}
    We first want to show $h : Z \to X \times Y$ exists.
    \begin{lxl}[resume]
        \item Suppose 
    \end{lxl}
    Next we want to show it is unique.
    \begin{lxl}[resume]
        \item
    \end{lxl}
    Thus there exists a unique $h$ that satisfies $\pi_{X\times Y\to X} \circ h = f$ and $\pi_{X\times Y\to Y} \circ h = g$.
\end{proof}
\begin{xx}
    Unfinished
\end{xx}

\subsection{}
\begin{q}
    Let $X_1, \ldots, X_n$ be sets. Show that the Cartesian product $\prod_{i=1}^n X_i$ is empty if and only if at least one of the $X_i$ is empty.
\end{q}

\begin{proof}
    We first show the forward direction.
    \begin{lxl}
        \item By lemma 3.5.11 If each $X_i$ is nonempty $\prod_{i=1}^n X_i$ is nonempty.
        \item The contrapositive states $\prod_{i=1}^n X_i$ is empty implies some $X_i$ is empty.
    \end{lxl}

    Let $k$ st. $X_k$ is empty.
        Suppose $a$ in $\prod_{i=1}^n X_i$.    
            Then $a_k \in X_k$.
            But $a_k \notin X_k$ since $X_k$ is empty.
            Thus, a contradiction!
        Therefore for all $a$, we know $a \notin \prod_{i=1}^n X_i$.
        Thus $\prod_{i=1}^n X_i = \emptyset$.
    So if $X_i$ for some $i$ is empty, $\prod_{i=1}^n X_i$ is empty. 


\end{proof}
\begin{xx}
    
\end{xx}

\subsection{}
\begin{q}
    Suppose that $I$ and $J$ are two sets, and for all $\alpha \in I$ let $A_{\alpha}$ be a set, and for all $\beta \in J$ let $B_{\beta}$ be a set. Show that
        \[
        \left(\bigcup_{\alpha \in I} A_{\alpha}\right) \cap \left(\bigcup_{\beta \in J} B_{\beta}\right) = \bigcup_{(\alpha, \beta) \in I \times J} \left(A_{\alpha} \cap B_{\beta}\right).
        \]
\end{q}

\begin{proof}
    We are proving set equality, so each element of $(\bigcup_{\alpha \in I} A_{\alpha}) \cap (\bigcup_{\beta \in J} B_{\beta})$ needs to be in $\bigcup_{(\alpha, \beta) \in I \times J} (A_{\alpha} \cap B_{\beta})$ and vice versa.
    \begin{lxl}
        \item Suppose $x \in (\bigcup_{\alpha \in I} A_{\alpha}) \cap (\bigcup_{\beta \in J} B_{\beta})$
        \begin{lxl}
            \item $x \in (\bigcup_{\alpha \in I} A_{\alpha})$
            \item $x \in (\bigcup_{\beta \in J} B_{\beta})$
            \item Suppose $(i,j) \in I \times J$ such that $x \in A_i$ and $x \in B_j$
            \begin{lxl}
                \item $x \in A_i$ and $x \in B_j$
                \item $x \in (A_i \cap B_j)$
                \item $i \in I$ and $j \in J$
                \item $x \in \bigcup_{(\alpha, \beta) \in I \times J} (A_{\alpha} \cap B_{\beta})$
            \end{lxl}
            \item $x \in \bigcup_{(\alpha, \beta) \in I \times J} (A_{\alpha} \cap B_{\beta})$ \why{MP}
        \end{lxl}
        \item $x \in (\bigcup_{\alpha \in I} A_{\alpha}) \cap (\bigcup_{\beta \in J} B_{\beta}) \implies x \in \bigcup_{(\alpha, \beta) \in I \times J} (A_{\alpha} \cap B_{\beta})$
    \end{lxl}
    The reverse direction:
    \begin{lxl}[resume]
        \item Suppose $x \in \bigcup_{(\alpha, \beta) \in I \times J} (A_{\alpha} \cap B_{\beta})$
        \begin{lxl}
            \item $x \in (A_{\alpha} \cap B_{\beta})$ for some $(\alpha, \beta) \in I \times J$ 
            \item Suppose $(i, j) \in I \times J$ and $x \in (A_{i} \cap B_{j})$
            \begin{lxl}
                \item $x \in (A_{i} \cap B_{j})$
                \item $x \in A_{i}$ and $x \in B_{j}$ 
            \end{lxl}
                \item $x \in (\bigcup_{\alpha \in I} A_{\alpha})$
                \item $x \in (\bigcup_{\beta \in J} B_{\beta})$
                \item $x \in (\bigcup_{\alpha \in I} A_{\alpha}) \cap (\bigcup_{\beta \in J} B_{\beta})$
        \end{lxl}
        \item $x \in \bigcup_{(\alpha, \beta) \in I \times J} (A_{\alpha} \cap B_{\beta}) \implies x \in (\bigcup_{\alpha \in I} A_{\alpha}) \cap (\bigcup_{\beta \in J} B_{\beta})$ 
    \end{lxl}
    Then each has all the elements of the other, and they are equal.
\end{proof}
\begin{xx}
    
\end{xx}
        
\begin{q}
    What happens if one interchanges all the union and intersection symbols here?
\end{q}

\begin{ans}
    
\end{ans}
    
\begin{proof}
    
\end{proof}
\begin{xx}
    
\end{xx}

\subsection{}
\begin{q}
    If $f : X \to Y$ is a function, define the graph of $f$ to be the subset of $X \times Y$ defined by $\{(x, f(x)) : x \in X\}$.
\end{q}

\begin{enumerate}
    \item \begin{q}
        Show that two functions $f : X \to Y$, $\tilde{f} : X \to Y$ are equal if and only if they have the same graph.
    \end{q}
    \begin{proof}
        We begin with the forward direction.
        \begin{lxl}
            \item Assume $f : X \to Y$ = $g : X \to Y$.
            \item The graph of $f$ = $\{(x, f(x)) : x \in X\}$.
            \item The graph of $g$ = $\{(x, g(x)) : x \in X\}$.
        \end{lxl}
        We need to now show these two sets are equal, by showing all the elements of one are contained in the other and vice versa.
        \begin{lxl}[resume]
            \item Suppose $a \in f$.
            \begin{lxl}
                \item $a \in \{(x, f(x)) : x \in X\}$
                \item $\exists x \in X \; a = (x, f(x))$ 
                \item Suppose $x \in X$ such that $a = (x, f(x))$
                \begin{lxl}
                    \item $a = (x, f(x))$
                    \item $f(x)=g(x)$
                    \item $(x, f(x)) = (x, g(x))$
                    \item $a = (x, g(x))$
                \end{lxl}
                \item $a \in \{(x, g(x)) : x \in X\}$
                \item $a \in g$
            \end{lxl}
            \item The other way is basically the same.
            \item $a \in f \implies a \in g$ and $a \in g \implies a \in f$ so they are equal.
        \end{lxl}
        Now the converse.
        We've got to show that the functions are the same, which mean they have the same domain and codomain, and each input has the same output.
        \begin{lxl}[resume]
            \item $\forall x \in X \; (\exists y \in Y$ such that $f(x)=y)$ and $(\forall y_1, y_2, f(x)=y_1$ and $f(x)=y_2 \implies y_1 = y_2)$
            \item Suppose $\{(x, f(x)) : x \in X\}$ = $\{(x, g(x)) : x \in X\}$.
            \item Let $x \in X$
            \begin{lxl}
                \item $(x, f(x))\in\{(x, f(x)) : x \in X\}$\why{by definition}
                \item $(x, f(x))\in\{(x, g(x)) : x \in X\}$\why{line 8}
                \item Suppose $(x, y)\in\{(x, g(x)) : x \in X\}$ and $f(x)\neq y$
                \begin{lxl}
                    \item $g(x) = y$\why{$(x,y)\in \{(x, g(x)) : x \in X\}$ and definition of graph}
                    \item $g(x) = f(x)$\why{$(x,f(x))\in \{(x, g(x)) : x \in X\}$ and definition of graph}
                    \item $f(x)=y$, a contradiction!
                \end{lxl}
                \item $f(x) = g(x)$
            \end{lxl}
            \item $\forall x\; f(x)=g(x)$
            \item $f=g$
            \item We already know $f$ and $g$ have the same domain and codomain.
        \end{lxl} 
        \begin{lxl}[resume]
            \item Thus they are equal!
        \end{lxl}
    \end{proof}
    \begin{xx}
        
    \end{xx}

    \item \begin{q}
        Conversely, if $G$ is any subset of $X \times Y$ with the property that for each $x \in X$, the set $\{y \in Y : (x, y) \in G\}$ has exactly one element (or in other words, $G$ obeys the vertical line test), show that there is exactly one function $f : X \to Y$ whose graph is equal to $G$.
    \end{q}

        
    \begin{proof}
        
    \end{proof}
    \begin{xx}
        
    \end{xx}
    \item \begin{q}
        Suppose we define a function $f$ to be an ordered triple $f = (X, Y, G)$, where $X, Y$ are sets, and $G$ is a subset of $X \times Y$ that obeys the vertical line test. We then define the domain of such a triple to be $X$, the codomain to be $Y$ and for every $x \in X$, we define $f(x)$ to be the unique $y \in Y$ such that $(x, y) \in G$. Show that this definition is compatible with Definition 3.3.1 in the sense that every choice of domain $X$, codomain $Y$, and property $P(x, y)$ obeying the vertical line test produces a function as defined here that obeys all the properties required of it in that definition, and is also similarly compatible with Definition 3.3.8.
    \end{q}
        
    \begin{proof}
        
    \end{proof}
    \begin{xx}
        
    \end{xx}

\end{enumerate}

\end{document}