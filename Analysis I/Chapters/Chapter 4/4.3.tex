\documentclass[../../main.tex]{subfiles}
\begin{document}
\exerciseset{4}{3}{Absolute value and exponentiation}

\addtocounter{subsection}{2}
\subsection{}
\begin{q}
    Let $x, y$ be rational numbers and let $n, m$ be natural numbers. 
\end{q}
\begin{enumerate}
    \addtocounter{enumi}{1}
    \item
    \begin{q}
        Suppose $n > 0$. Then we have $x^n = 0$ if and only if $x = 0$.
    \end{q}
    
    \begin{proof}
        We prove by induction.

        Suppose $n = 1$ as a base case.
        Since $x = x^1$, $x = 0 \iff x^n = 0$.
        
        Now suppose $x = 0 \iff x^n = 0$. 
        We have $x^{n+1} = x^n * x$.
        Now $x = 0$ or $x \neq 0$.
        
        If $x = 0$, $x^n = 0$ by inductive assumption. 
        Then $x^{n+1} = 0 * 0 = 0$, so $x = 0 \implies x^{n+1} = 0$.

        If instead $x \neq 0$, $x^n \neq 0$ we have $x^{n+1} \neq 0$, since the product of two nonzero rationals is a nonzero rational. (Extension of 4.1.8 that I haven't proved)

        Thus we have $x^{n+1} = 0 \iff x = 0$, closing the induction, and so for all $x$, for all $n > 0$, we have $x^n = 0$ if and only if $x = 0$.
    \end{proof}
\end{enumerate}\begin{q}
    
\end{q}

\end{document}