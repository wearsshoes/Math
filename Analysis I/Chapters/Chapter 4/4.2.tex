\documentclass[../../main.tex]{subfiles}
\begin{document}
\exerciseset{4}{2}{The rationals}

\subsection{}
\begin{q}
    Show that the definition of equality for the rational numbers is reflexive, symmetric, and transitive. (Hint: for transitivity, use Corollary 4.1.9.)
\end{q}

\begin{proof}
    We first prove reflexivity, 
    which holds if $a // b = a // b$, $b \neq 0$.
    $ab=ab$ because reflexivity holds for the integers, 
    and $ab=ab \iff a // b = a // b$ by definition.

    Next we tackle symmetry, 
    which holds if  $a // b = c // d \iff c // d = a // b$. 
    We have $a // b = c // d \iff ad = bc$, $b \neq 0$ and $d \neq 0$.
    We also have $c // d = a // b \iff cb = da$. 
    By commutativity of multiplication on the integers $ad = da$ and $bc = cb$.
    So by symmetry for the integers $ad = bc \iff cb = ad$, 
    and thus $a // b = c // d \iff c // d = a // b$.     

    Finally for transitivity, which holds if $a // b = c // d$ and $c // d = e // f$ implies $a // b = e // f$.
    We have $a // b = c // d \iff ad = bc$,
    we also have $c // d = e // f \iff cf = de$,
    where $b, d, f$ are all nonzero.
    By the cancellation rule (Lemma 4.1.9), $adf = bcf$.
    Also by 4.1.9, $bcf = bde$.
    By transitivity for the integers, $adf = bde$, 
    and by 4.1.9 again $af = be$,
    so $a // b = e // f$ as desired.
\end{proof}
\pagebreak

\addtocounter{subsection}{1}
\subsection{}
\begin{q}
    Prove the remaining components of Proposition 4.2.4.    
\end{q}

I will only prove a subset of these, using $x = a // b$ , $y = c // d$ , and $z = e // f$ for integers $a, c, e$ and nonzero integers $b, d, f$.
\begin{enumerate}
    \item
        \begin{q}
            $(x y)z = x(yz)$
        \end{q}
        \begin{proof}
            \begin{align*}
                ((a // b)(c // d))(e // f) &= \\
                (ac // bd)(e // f) &= \\
                (ace // bdf) & \\
                (a // b)((c // d)(e // f)) &= \\
                (a // b)(ce // df) &= \\
                (ace // bdf) &\\
            \end{align*}
        \end{proof}
    \item 
        \begin{q}
            $x(y + z) = x y + x z$
        \end{q}
        \begin{proof}
            \begin{align*}
                (a // b)((c // d) + (e // f)) &= \\
                (a // b)((cf + de) // (df)) &= \\
                (a(cf + de) // bdf) &= \\
                (acf + ade) // bdf & \\
                (a // b)(c // d) + (a // b)(e // f) & =\\
                (ac // bd) + (ae // bf) &= \\
                (acbf + aebd) // (bdbf) &= \\
                (acbf + aebd) // (bdbf) &= \\
                ((acf + aed) // bdf) * (b // b) &= \\
                ((acf + aed) // bdf) * (1 // 1) &= \\
                ((acf + aed) // bdf) * 1 &= \\
                ((acf + aed) // bdf) &= \\
                ((acf + ade) // bdf) & \\
            \end{align*}
        \end{proof} 
\end{enumerate}

\addtocounter{subsection}{1}
\subsection{}
\begin{q}
    Proposition 4.2.9 (Basic properties of order on the rationals) Let $x, y,z$ be rational
numbers. Then the following properties hold.
\end{q}

Let $x = a/b$, $y = c/d$, and $z = e/f$ for integers $a, b, c, d, e, f$ where $b, d, f$ are nonzero.
\begin{enumerate}
    \item
    \begin{q}
         (Order trichotomy) Exactly one of the three statements $x = y$, $x < y$, or $x > y$ is true.
    \end{q}
    \begin{proof}
        $x < y \implies x \neq y$

        $x > y \implies x \neq y$

        Then by contrapositive $x = y$ implies both $x < y$ and $x > y$ are false.

        $x > y \iff x-y$ is positive

        $x < y \iff x-y$ is negative

        By the trichotomy of rationals $x-y$ cannot be both positive and negative. So $x > y$ and $x < y$ are exclusive.
    \end{proof}
    \item 
    \begin{q}
        (Order is antisymmetric) One has $x < y$ if and only if $y > x$.
    \end{q}
    \begin{proof}
        We have two directions to prove.

        Suppose $x < y$. 
        By def of order on rationals $x-y$ is negative, i.e 
        $(ad - bc) / bd$ is negative.
        Then by def of negative on rationals $-(ad - bc)$ and $bd$ are both positive integers.
        $-(ad - bc) = (cb - da)$ by negation and commutativity.
        Therefore $(cb - da) / bd$ is a positive rational, which is equal to $(y-x)$.
        Thus, $y > x$.

        Suppose $y > x$. 
        Then $y-x$ is positive, i.e.
        $(cb - da) / db$ is positive.
        Then $(cb - da)$ and $(db)$ are both positive integers.
        If $(cb - da)$ is positive, its negation $-(cb - da)$ is negative.
        We know $-(cb - da) = (ad - bc)$ by negation and commutativity.
        Then $(ad - bc) / db$ is a negative rational. 
        Then so is $(x-y)$.
        Thus, $x < y$.
    \end{proof}
    \item 
    \begin{q}
        (Order is transitive) If $x < y$ and $y < z$, then $x < z$.
    \end{q}
    \begin{proof}[Lemma]
        First we need to prove $(-a)/b = a/(-b)$, so we can write the proof without loss of generality.
        \begin{lxl}
            \item $(-a)/b = a/(-b) \iff -a(-b) = ab$. 
            \item $-a(-b)=$
            \item $(-1)(-1)(ab)=$
            \item $(0$---$1)(0$---$1)(ab)=$
            \item $((0*0 + 1*1)-(0*1+1*0))(ab)=$
            \item $((0+1)-(0+0))(ab)=$
            \item $(1-0)(ab)=$
            \item $1(ab)=$
            \item $ab$.
            \item Thus $(-a)/b = a/(-b)$ as desired. 
        \end{lxl}
    \end{proof}
    \begin{proof}[Lemma]
        We wish to prove $a/b < c/d \iff ad < bc$, and $b, d \neq 0$.
        From above, we can assume without loss of generality that $b > 0$ and $d > 0$ (otherwise choose $-a/-b$ and $-c/-d$). 
        Suppose $a/b < c/d$.
        We know $a/b < c/d \iff (a/b - c/d)$ is a negative rational number.
        By definition of subtration $(ad - bc) / bd$ is negative. 
        By our generality assumption, $bd$ is positive, so $(ad-bc)$ is negative.
        Suppose $ad = bc$, then $(ad - bc) = 0$, so by contradiction $ad \neq bc$. 
        There must be some $m$ such that $(ad - bc) + m = 0$, 
        so there exists some nonzero $m$ such that $ad+m=bc$.
        Thus by the previous two statements, $ad < bc$.
        Suppose $ad < bc$.
        Then $ad+m=bc$, $m \neq 0$, and $ad - bc = 0 - m$, or $(ad-bc)$ is negative. 
        Since we have assumed $bd$ is positive, the rational $(ad - bc) / bd$ is negative. 
        $(ad - bc) / bd = (a/b - c/d)$, so the latter is negative too.
        Thus $a/b < c/d$.
        
    \end{proof}
    \begin{proof}
        Without loss of generality, assume $b, d, f > 0$. 
        (If $b>0$, 
        then set $a'=a,b'=b$; 
        otherwise set $a'=-a, b'=-b$, 
        and note that in either case $x=a'/b'$ and $b'>0$.) 
        Likewise for $y, z$.
        \begin{lxl}
            \item Suppose $x < y$ and $y < z$.
            \item Then $ad < bc$ by above lemma.
            \item Likewise, $cf < de$. 
            \item $adf < bcf$ since we assume $f$ is positive, and Lemma 4.1.11c (positive multiplication preserves order on integers).
            \item $bcf < deb$ since we assume $b$ is positive.
            \item $adf < deb$ by transitivity of order on integers.
            \item $af < eb$ since we assume $d$ is positive.
            \item $x < z$.
        \end{lxl}
    \end{proof}
    \item 
        \begin{q}
           (Addition preserves order) If $x < y$, then $x + z < y + z$. 
        \end{q}
    \item 
        \begin{q}
            (Positive multiplication preserves order) If $x < y$ and $z$ is positive, then $x z < yz$.
        \end{q}
\end{enumerate}

\end{document}